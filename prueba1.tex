\documentclass{article}

\usepackage[utf8]{inputenc}
\usepackage{amsmath}
\usepackage[spanish]{babel}

\title{Apuntes de programacion lineal}

\author{Hector Roldan}
 



\begin{document}
\maketitle
\tableofcontents

\section{Introduccion}
\label{sec:introduccion}


La forma estandar de un problema de programacion lineal es:


Dados una matriz $A$ y vectores $b,c$, maximizar $c^Tx$ sujeto a
$Ax\leq b$.

Podemos pasar el problema a la forma Simplex usando variables de
holgura, para pasar el problema a la forma:

Dada una matriz $A_1$ y vectores $b,c_1$, maximizar $c_1^Tx$ sujeto a $A_1x=b$.

\subsection{Miscelanea}
\label{sec:miscelanea}


\begin{tabular}{c|c|c}
  &A&B\\
  \hline
  Maquina 1&1&2\\
  Maquina 2&1&1
\end{tabular}
\begin{equation}
  \label{eq:1}
  N=
  \begin{pmatrix}
    0&1&2\\
    3&-1&5\\
    a&b_1&c^2
  \end{pmatrix}
  \begin{pmatrix}
    x_{1,1}&x_{1,2}\\
    x_{2,1}&x_{2,2}\\
    x_{3,1}&x_{3,2}
  \end{pmatrix}
\end{equation}

\end{document}
